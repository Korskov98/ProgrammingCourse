\documentclass[12pt,a4paper]{report}
\usepackage[utf8]{inputenc}
\usepackage[russian]{babel}
\usepackage[OT1]{fontenc}
\usepackage{amsmath}
\usepackage{amsfonts}
\usepackage{amssymb}
\usepackage{graphicx}
\usepackage{cmap}					% поиск в PDF
\usepackage{mathtext} 				% русские буквы в формулах
%\usepackage{tikz-uml}               % uml диаграммы

% TODOs
\usepackage[%
  colorinlistoftodos,
  shadow
]{todonotes}

% Генератор текста
\usepackage{blindtext}

%------------------------------------------------------------------------------

% Подсветка синтаксиса
\usepackage{color}
\usepackage{xcolor}
\usepackage{listings}
 
 % Цвета для кода
\definecolor{string}{HTML}{B40000} % цвет строк в коде
\definecolor{comment}{HTML}{008000} % цвет комментариев в коде
\definecolor{keyword}{HTML}{1A00FF} % цвет ключевых слов в коде
\definecolor{morecomment}{HTML}{8000FF} % цвет include и других элементов в коде
\definecolor{captiontext}{HTML}{FFFFFF} % цвет текста заголовка в коде
\definecolor{captionbk}{HTML}{999999} % цвет фона заголовка в коде
\definecolor{bk}{HTML}{FFFFFF} % цвет фона в коде
\definecolor{frame}{HTML}{999999} % цвет рамки в коде
\definecolor{brackets}{HTML}{B40000} % цвет скобок в коде
 
 % Настройки отображения кода
\lstset{
language=C, % Язык кода по умолчанию
morekeywords={*,...}, % если хотите добавить ключевые слова, то добавляйте
 % Цвета
keywordstyle=\color{keyword}\ttfamily\bfseries,
stringstyle=\color{string}\ttfamily,
commentstyle=\color{comment}\ttfamily\itshape,
morecomment=[l][\color{morecomment}]{\#}, 
 % Настройки отображения     
breaklines=true, % Перенос длинных строк
basicstyle=\ttfamily\footnotesize, % Шрифт для отображения кода
backgroundcolor=\color{bk}, % Цвет фона кода
%frame=lrb,xleftmargin=\fboxsep,xrightmargin=-\fboxsep, % Рамка, подогнанная к заголовку
frame=tblr
rulecolor=\color{frame}, % Цвет рамки
tabsize=3, % Размер табуляции в пробелах
showstringspaces=false,
 % Настройка отображения номеров строк. Если не нужно, то удалите весь блок
numbers=left, % Слева отображаются номера строк
stepnumber=1, % Каждую строку нумеровать
numbersep=5pt, % Отступ от кода 
numberstyle=\small\color{black}, % Стиль написания номеров строк
 % Для отображения русского языка
extendedchars=true,
literate={Ö}{{\"O}}1
  {Ä}{{\"A}}1
  {Ü}{{\"U}}1
  {ß}{{\ss}}1
  {ü}{{\"u}}1
  {ä}{{\"a}}1
  {ö}{{\"o}}1
  {~}{{\textasciitilde}}1
  {а}{{\selectfont\char224}}1
  {б}{{\selectfont\char225}}1
  {в}{{\selectfont\char226}}1
  {г}{{\selectfont\char227}}1
  {д}{{\selectfont\char228}}1
  {е}{{\selectfont\char229}}1
  {ё}{{\"e}}1
  {ж}{{\selectfont\char230}}1
  {з}{{\selectfont\char231}}1
  {и}{{\selectfont\char232}}1
  {й}{{\selectfont\char233}}1
  {к}{{\selectfont\char234}}1
  {л}{{\selectfont\char235}}1
  {м}{{\selectfont\char236}}1
  {н}{{\selectfont\char237}}1
  {о}{{\selectfont\char238}}1
  {п}{{\selectfont\char239}}1
  {р}{{\selectfont\char240}}1
  {с}{{\selectfont\char241}}1
  {т}{{\selectfont\char242}}1
  {у}{{\selectfont\char243}}1
  {ф}{{\selectfont\char244}}1
  {х}{{\selectfont\char245}}1
  {ц}{{\selectfont\char246}}1
  {ч}{{\selectfont\char247}}1
  {ш}{{\selectfont\char248}}1
  {щ}{{\selectfont\char249}}1
  {ъ}{{\selectfont\char250}}1
  {ы}{{\selectfont\char251}}1
  {ь}{{\selectfont\char252}}1
  {э}{{\selectfont\char253}}1
  {ю}{{\selectfont\char254}}1
  {я}{{\selectfont\char255}}1
  {А}{{\selectfont\char192}}1
  {Б}{{\selectfont\char193}}1
  {В}{{\selectfont\char194}}1
  {Г}{{\selectfont\char195}}1
  {Д}{{\selectfont\char196}}1
  {Е}{{\selectfont\char197}}1
  {Ё}{{\"E}}1
  {Ж}{{\selectfont\char198}}1
  {З}{{\selectfont\char199}}1
  {И}{{\selectfont\char200}}1
  {Й}{{\selectfont\char201}}1
  {К}{{\selectfont\char202}}1
  {Л}{{\selectfont\char203}}1
  {М}{{\selectfont\char204}}1
  {Н}{{\selectfont\char205}}1
  {О}{{\selectfont\char206}}1
  {П}{{\selectfont\char207}}1
  {Р}{{\selectfont\char208}}1
  {С}{{\selectfont\char209}}1
  {Т}{{\selectfont\char210}}1
  {У}{{\selectfont\char211}}1
  {Ф}{{\selectfont\char212}}1
  {Х}{{\selectfont\char213}}1
  {Ц}{{\selectfont\char214}}1
  {Ч}{{\selectfont\char215}}1
  {Ш}{{\selectfont\char216}}1
  {Щ}{{\selectfont\char217}}1
  {Ъ}{{\selectfont\char218}}1
  {Ы}{{\selectfont\char219}}1
  {Ь}{{\selectfont\char220}}1
  {Э}{{\selectfont\char221}}1
  {Ю}{{\selectfont\char222}}1
  {Я}{{\selectfont\char223}}1
  {і}{{\selectfont\char105}}1
  {ї}{{\selectfont\char168}}1
  {є}{{\selectfont\char185}}1
  {ґ}{{\selectfont\char160}}1
  {І}{{\selectfont\char73}}1
  {Ї}{{\selectfont\char136}}1
  {Є}{{\selectfont\char153}}1
  {Ґ}{{\selectfont\char128}}1
  {\{}{{{\color{brackets}\{}}}1 % Цвет скобок {
  {\}}{{{\color{brackets}\}}}}1 % Цвет скобок }
}
 
 % Для настройки заголовка кода
\usepackage{caption}
\DeclareCaptionFont{white}{\color{сaptiontext}}
\DeclareCaptionFormat{listing}{\parbox{\linewidth}{\colorbox{сaptionbk}{\parbox{\linewidth}{#1#2#3}}\vskip-4pt}}
\captionsetup[lstlisting]{format=listing,labelfont=white,textfont=white}
\renewcommand{\lstlistingname}{Код} % Переименование Listings в нужное именование структуры


%------------------------------------------------------------------------------

\author{Корсков А. В.}
\title{Программирование}
\begin{document}
%\listoftodos
\maketitle
\chapter{Основные конструкции языка}
%############################################################
\section{Задание 1}
\subsection{Задание}
%\todo[inline]{Начать можно с этого}
В морской миле 2000 ярдов или 6000 футов. Задано некоторое расстояние в футах, например, 9139 футов. Вывести то же расстояние в милях, ярдах и футах, например, 9139 = 1 миля 1046 ярдов и 1 фут.
\subsection{Теоретические сведения}
Для реализации данного алгоритмы были использованы функции стандартной библеотеки для ввода и вывода информации. Также для передачи значений из функции мы использовали указатели. В ходе доработки было принято решение использовать структуры.

Морская миля — единица измерения расстояния, применяемая в мореплавании и авиации. Ярд (англ. yard) — британская и американская единица измерения расстояния. Фут — единица измерения длины в английской системе мер. Мы знаем, что в 1 миле содержится 2000 ярдов или 6000 футов. А в 1 ярде содержится 3 фута.
\subsection{Проектирование}
В функции main вызывается функция convert\_UI. Взаимодействие с пользователем реализовано в фунции convert\_UI. Она считывает ввёдные пользователем значения из консоли, вызывает функцию convert, а после выполнения функции convert, выводит результат в консоль. Для реализации алгоритма мы выделили функцию convert. Функция принимает один указатель на структуру. Сначала в функции вычисляется колличество миль. Далее находится колличество футов без миль. На следующем шаге вычисляется колличество ярдов. И на последнем шаге находится колличество футов. Функция ничего не возвращает, так как для передачи значения мы пользуемся указателями.
\subsection{Описание тестового стенда и методики тестирования}
Для решения задачи использовалась вертуальная машина с операционной системой Debian 8.1, в которой установлен QtCreator 3.5.0. 
Пользователь может сам ввести его данные или воспользоваться автоматическими тестами, запустив их отдельно.
\subsection{Тестовый план и результаты тестирования}
В программе предусмотрены тесты, которые проверяют правильность исполнения программы. В тесте вызывается функция, ей передаётся колличество футов, равное 9139. Ожидается, что результатом работы программы будет колличество футов равное 1, колличество ярдов равное 1046 и колличество миль равное 1. Далее каждая из трёх переменных проверяется на соответствие с ожидаемым правильным результатом. При вызове теста ошибок не обнаружено.
\subsection{Выводы}
В ходе работы мы научились структурировать проект, разбивая задачи на мелкие подзадачи. Получили опыт создания тестов. Также мы научились работать с системой контроля версий и приобрели навыки работы с указателями. Также получили опыт работы со структурами.
\subsection*{Листинги}
\lstinputlisting[]
{../sources/home-work/app/main.c}

\lstinputlisting[]
{../sources/home-work/app/convert_UI.c}

\lstinputlisting[]
{../sources/home-work/lib/convert.c}

%\todo[inline]{Не забыть вставить все исходники}
%############################################################

\section{Задание 2}
\subsection{Задание}
Заданы три целых числа: a, b, c. Определить, могут ли они быть длинами сторон треугольника, и если да, определить, является ли он равнобедренным либо равносторонним.
\subsection{Теоритические сведения}
Треугольник существует только тогда, когда сумма любых двух его сторон больше третьей. У равнобедренного треугольника две стороны равны. У равностороннего все стороны равны.
Для реализации данного алгоритмы были использованы функции стандартной библеотеки для ввода и вывода информации. Также использовалась конструкция if...else. 
\subsection{Проектирование}
В функции main вызывается функция check\_UI. Взаимодействие с пользователем реализовано в фунции check\_UI. Она считывает ввёдные пользователем значения из консоли, вызывает функцию check, а после выполнения функции check, выводит результат в консоль. Для реализации алгоритма мы выделили функцию check. Функция принимает три переменные типа int. Каждая из этих трёх переменных задаёт одну из сторон треугольника. Сначала входные данные проверяются на корректность. Далее проверяем возможен ли данный треугольник или нет. Затем смотрим является ли данный треугольник равнобедренним. Если это условие выполняется, проверяем будет ли данный треугольник равносторонним.
\subsection{Описание тестового стенда и методики тестирования}
Для решения задачи использовалась вертуальная машина с операционной системой Debian 8.1, в которой установлен QtCreator 3.5.0. 
Пользователь может сам ввести его данные или воспользоваться автоматическими тестами, запустив их отдельно.
\subsection{Тестовый план и результаты тестирования}
В программе предусмотрены тесты, которые проверяют правильность исполнения программы. В тесте вызывается функция, ей передаётся три стороны треугольника, которые все равны 3. Ожидается, что результатом работы программы будет цифра 3, означающая, что треугольник равносторонний. Далее полученное число сверяется с ожидаемым правильным ответом. При вызове теста ошибок не обнаружено.
\subsection{Выводы}
В ходе работы мы научились работать с конструкцией if...else. Получили опыт создания вложенных операторов if...else.
\lstinputlisting[]
{../sources/home-work/app/check_UI.c}

\lstinputlisting[]
{../sources/home-work/lib/check.c}

%############################################################
\chapter{Циклы}
\section{Задание 1}
\subsection{Задание}
Найти число, полученное из данного выбрасыванием нечётных цифр.
\subsection{Теоритические сведения}
Для решения данной задачи ипользовалась конструкция if..else. А также цикл с предусловием while и функции стандартной библеотеки для ввода и вывода информации.
\subsection{Проектирование}
В функции main вызывается функция removal\_UI. Взаимодействие с пользователем реализовано в фунции removal\_UI. Она считывает ввёдные пользователем значения из консоли, вызывает функцию removal, а после выполнения функции removal, выводит результат в консоль. Для реализации алгоритма мы выделили функцию removal. Функция принимает одно значение типа int. Это значение задаёт число, в котором нужно удалить цифры. Далее пока число не будет равняться 0 оно будет делиться на 10. От числа отрезается последняя цифра и если она чётная записать её в результат.
\subsection{Описание тестового стенда и методики тестирования}
Для решения задачи использовалась вертуальная машина с операционной системой Debian 8.1, в которой установлен QtCreator 3.5.0. 
Пользователь может сам ввести его данные или воспользоваться автоматическими тестами, запустив их отдельно.
\subsection{Тестовый план и результаты тестирования}
В программе предусмотрены тесты, которые проверяют правильность исполнения программы. В тесте вызывается функция, ей передаётся число равное 2597. Ожидается, что результатом работы программы будет цифра 2.Далее полученное число сверяется с ожидаемым правильным ответом. При вызове теста ошибок не обнаружено.
\subsection{Выводы}
В этом задании мы научились работать с циклом while.
\lstinputlisting[]
{../sources/home-work/app/removal_UI.c}

\lstinputlisting[]
{../sources/home-work/lib/removal.c}
\chapter{Матрицы}
\section{Задание 1}
\subsection{Задание}
Латинским квадратом порядка n называется квадратная таблица размером n x n, каждая строка и каждый столбец которой содержат все числа от 1 до n. Проверить, является ли заданная целочисленная матрица латинским квадратом.
\subsection{Теоритические сведения}
Двумерный массив является латинским квадратом, если во всех его строках и столбцах есть все цифры от 1 до n.

Для решения данной задачи ипользовалась конструкция if..else, цикл for функции стандартной библеотеки для ввода и вывода информации. Также в задаче понадобилось использовать двумерный массив и выделять для него память.
\subsection{Проектирование}
В функции main вызывается функция square\_UI. Взаимодействие с пользователем реализовано в фунции square\_UI. Она считывает ввёдные пользователем значения из файла, вызывает функцию square, а после выполнения функции square, выводит результат в файл. Для реализации алгоритма мы выделили функцию square. Функция принимает одно значение типа int и указатель на двумерный массив. Далее мы проходим по всем столбцам и строкам и ищем все цифры от 1 до n, которое мы передали функции square. Если какой-либо цифры нет мы выходим из функции с помощью переменной flag. 
\subsection{Описание тестового стенда и методики тестирования}
Для решения задачи использовалась вертуальная машина с операционной системой Debian 8.1, в которой установлен QtCreator 3.5.0. 
Пользователь может сам ввести его данные или воспользоваться автоматическими тестами, запустив их отдельно.
\subsection{Тестовый план и результаты тестирования}
В программе предусмотрены тесты, которые проверяют правильность исполнения программы. В тесте вызывается функция, ей передаётся двумерный массив размером 2x2. Ожидается, что результатом работы программы будет цифра 1. Далее полученный результат сверяется с ожидаемым правильным ответом. При вызове теста ошибок не обнаружено.
\subsection{Выводы}
В этом задании мы научились создавать двумерный массив, выделять под него память, а также манипулировать с его содержимым.
\lstinputlisting[]
{../sources/home-work/app/square_UI.c}

\lstinputlisting[]
{../sources/home-work/lib/square.c}
\chapter{Строки}
\section{Задание 1}
\subsection{Задание}
Часто встречающаяся ошибка начинающих наборщиков – дважды записанное слово. Обнаружить и исправить такие ошибки в тексте.
\subsection{Теоритические сведения}
Для решения данной задачи ипользовалась конструкция if..else, циклы for, while и функции стандартной библеотеки для ввода и вывода информации. Также в задаче понадобилось использовать функции стандартной библиотеки для работы со строками.
\subsection{Проектирование}
В функции main вызывается функция removing\_words\_UI. Взаимодействие с пользователем реализовано в фунции removing\_words\_UI. Она считывает ввёдные пользователем значения из файла, вызывает функцию removing\_words. Для реализации алгоритма мы выделили функцию removing\_words. Функция принимает одно значение типа массив char. Мы отрезаем каждое слово до пробела и сравниваем со следующим словом и если слова различны записываем данное слово.
\subsection{Описание тестового стенда и методики тестирования}
Для решения задачи использовалась вертуальная машина с операционной системой Debian 8.1, в которой установлен QtCreator 3.5.0. 
\subsection{Тестовый план и результаты тестирования}
Программе передаётся строка "ui ui test test". Ожидается, что результатом работы программы будет строка "ui test". При вызове теста ошибок не обнаружено.
\subsection{Выводы}
В этом задании мы научились работать со строками.
\lstinputlisting[]
{../sources/home-work/app/removing_words_UI.c}

\lstinputlisting[]
{../sources/home-work/lib/removing_words.c}
\chapter{Инкапсуляция}
\section{Задание 1}
\subsection{Задание}
Реализовать класс ВЕКТОР (многомерный). Требуемые методы: конструктор, деструктор, сложение, вычитание, копирование, скалярное произведение, умножение на число, модуль.
\subsection{Теоритические сведения}
Два вектора имеют координаты $ x_1, y_1, x_2, y_2 $. В результате их сложения получится вектор с координатами $ x_1+x_2 $ и $ y_1+y_2 $. В результате их вычитания получится вектор с координатами $ x_1-x_2 $ и $ y_1-y_2 $. В результате их скалярного произведения получится число равное $ x_1 * x_2 + y_1 * y_2 $. В результате умножения вектора на число z получится вектор с координатами $ x_1*z $ и $ y_1*z $. В результате взятия модуля получится число равное $ \sqrt{x_1*x_1+y_1*y_1} $. 

Для решения данной задачи мы создали специальный класс. Его методы будут задавать действия над векторами.
\subsection{Проектирование}
В функции main вызывается функция works\_with\_vectors. Взаимодействие с пользователем реализовано в фунции works\_with\_vectors. С помощью неё пользователь может производить действия над векторами. Для реализации алгоритма мы выделили класс vectors. Его методы будут задавать действия над векторами. А также будут два поля x и y задающие координаты.
\subsection{Описание тестового стенда и методики тестирования}
Для решения задачи использовалась вертуальная машина с операционной системой Debian 8.1, в которой установлен QtCreator 3.5.0. 
\subsection{Тестовый план и результаты тестирования}
В программе предусмотрены тесты, которые проверяют правильность исполнения программы. В тесте проверяются все методы класса. Для теста сложения задаются два вектора с координатами (3,4) и (5,6). Ожидается, что в результате у первого вектора будут координаты (8,10). Для теста вычитания задаются два вектора с координатами (5,6) и (3,4). Ожидается, что в результате у первого вектора будут координаты (2,2). Для теста скалярного произведения задаются два вектора с координатами (3,4) и (5,6). Ожидается, что результатом выполнения программы будет число равное 39. Для теста умножения вектора на число задаётся вектор с координатами (3,4) и число 10. Ожидается, что в результате у первого вектора будут координаты (30,40). Для теста взятия модуля задаётся вектор с координатами (3,4). Ожидается, что результатом выполнения программы будет число равное 10. Все тесты проходят успешно.
\subsection{Выводы}
В этом задании мы научились создавать классы. Задавать методы и поля класса. А также производить действия над объектами одного класса.
\lstinputlisting[]
{../sources/home-work/cppapp/work_with_vectors.cpp}

\lstinputlisting[]
{../sources/home-work/cpplib/vectors.cpp}
Также в рамках задания были переписаны все программы, которые были сделаны на c, на c++.
\subsection{Задача 1}
\lstinputlisting[]
{../sources/home-work/cpplib/convert_class.cpp}
\subsection{Задача 2}
\lstinputlisting[]
{../sources/home-work/cpplib/check_class.cpp}
\subsection{Задача 3}
\lstinputlisting[]
{../sources/home-work/cpplib/removal_class.cpp}
\subsection{Задача 4}
\lstinputlisting[]
{../sources/home-work/cpplib/square_class.cpp}
\subsection{Задача 5}
\lstinputlisting[]
{../sources/home-work/cpplib/removing_words_class.cpp}
\end{document}
